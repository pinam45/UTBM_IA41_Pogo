\documentclass[article, backcover, french, nodocumentinfo]{upmethodology-document}
\include{packages}
% For more information about UPmethodology: https://www.ctan.org/pkg/upmethodology

%=======================================================================================================
%=============================================== Informations ==========================================
%=======================================================================================================

%%% Document Information and Declaration
\declaredocument{Jeu tactique: Pogo}{Projet d'IA41}{UTBM\_IA41\_Pogo\_P17}

%%% Abstract and Key-words
\setdocabstract[french]{Projet UTBM de l'UV IA41 du semestre de printemps 2017}
\setdockeywords[french]{UTBM, IA41, Pogo}

%%% Document Authors and Validators
\addauthorvalidator*[julien.barbier@utbm.fr]{Julien}{Barbier}{Étudiant en INFO02}
\addauthorvalidator*[kadir.ercin@utbm.fr]{Kadir}{Ercin}{Étudiant en INFO02}
\addauthorvalidator*[maxime.pinard@utbm.fr]{Maxime}{Pinard}{Étudiant en INFO02}

%%% Informed People
\addinformed*[fabrice.lauri@utbm.fr]{Fabrice}{Lauri}{Professeur de l'UV IA41}

%%% Copyright and Publication Information
\setcopyrighter{Julien Barbier, Kadir Ercin et Maxime Pinard}
\setpublisher{Julien Barbier, Kadir Ercin et Maxime Pinard}
\setprintingaddress{France}

%%% Version
\incversion{\makedate{\the\day}{\the\month}{\the\year}}{Initial version.}{\upmpublic}

%=======================================================================================================
%================================================== Configs ============================================
%=======================================================================================================

% Change Front Page Layout
%\setfrontcover{modern} % modern or classic

% Change Illustration Picture
%\setfrontillustration[1.3]{figures/figure}

% Source code formatting
\upmcodelang{cpp} % uml, java or cpp

% Prevent page breaks in paragraphs
\predisplaypenalty=1000
\postdisplaypenalty=1000
\clubpenalty=1000

% Minimal space required in the bottom margin not to move the title on the next page
%\renewcommand{\bottomtitlespace}{.1\textheight}

% Links config, especialy for the table of contents
\hypersetup{
    colorlinks=true,
    linkcolor=black,
    urlcolor=blue,
    linktoc=all
}

% French language config
%\frenchbsetup{StandardLayout=true,ReduceListSpacing=false,CompactItemize=false}

%=======================================================================================================
%================================================= Functions ===========================================
%=======================================================================================================

%Paragraph with line break
\newcommand{\p}[1]{\paragraph{#1\\}}

% Function to print a warning sign
\newcommand{\dangersign}[1][2.5ex]
	{\renewcommand{\stacktype}{L}
		{\scaleto{\stackon[1pt]{\color{red}$\triangle$}{\fontsize{4pt}{4pt}\selectfont !}}{#1}}}

% Definition of some dt/dx/dy shortcuts for integrals
\newcommand{\dt}
{\;\mathrm{d}\,t}

\newcommand{\dx}
{\;\mathrm{d}\,x}

\newcommand{\dy}
{\;\mathrm{d}\,y}

% Definition of \Witem for 'itemize' environment with a warning sign
\newcommand{\Witem}
{\item[\dangersign{}]}

% Definition of a Max function shortcut
\newcommand{\Max}[2][ ]
{\underset{#1}{\text{Max}}\,#2}

% Definition of a Min function shortcut
\newcommand{\Min}[2][ ]
{\underset{#1}{\text{Min}}\,#2}


\begin{document}
	\upmdocumentsummary{}
	\upmdocumentauthors{}
	%\upmdocumentvalidators{}
	\upmdocumentinformedpeople{}
	\upmpublicationpage{}
	\thispagestyle{empty}
	\tableofcontents{}
	\newpage{}
	\lstlistoflistings{}
	\listoffigures{}
	\newpage{}

	\section{Sujet, règles du jeu}
		\subsection{Présentation}
			Le but de ce projet est de réaliser une IA capable de jouer au jeu de Pogo. Pour cela l'algorithme \textbf{MinMax} et son optimisation \textbf{AlphaBeta} seront utilisés. Il doit aussi être possible de tester l'IA. Le langage utilisé pour réaliser le projet est le C++ norme 2011.
		\subsection{Règles}
			\p{Matériel}
				\begin{itemize}
					\item Plateau de 9 cases, dimensions 3 $\times$ 3
					\item 6 pions de couleurs différentes pour chaque joueur
				\end{itemize}
			\p{But du jeu}
				Être le premier à avoir une pièce de sa couleur au sommet de chaque pile.
			\p{Position initiale}
				La position initiale est illustrée avec notre interface utilisateur sur la figure \ref{fig:état_initial}.
				\begin{figure}[H]
					\centering
					\includegraphics[width=0.7\textwidth]{figures/ConsoleUI}
					\caption{Plateau: État initial du plateau de jeu}
					\label{fig:état_initial}
				\end{figure}
			\p{Description du jeu}
				A tour de rôle, chaque joueur effectue un déplacement qui doit obéir aux règles suivantes:
				\begin{itemize}
					\item Un joueur peut déplacer au choix, une, deux ou trois pièces, qui doivent déjà être empilées sur la même case.
					\item La pièce au sommet de l’ensemble déplacé doit être de sa couleur. Les pièces de dessous peuvent appartenir à l’autre joueur.
					\item L’ensemble déplacé doit bouger du même nombre de cases que le nombre d’éléments dont il est constitué.
				\end{itemize}
				Les pièces peuvent être déplacées dans toutes les directions, mais pas en diagonale. Le déplacement d’une pile de deux pièces peut se faire en coude. Celui d’une pile de trois pièces peut décrire un ou deux coudes. Peu importe si la case d’arrivée est occupée ou pas. Les cases occupées n’empêchent pas le passage des pièces. La hauteur des piles n’est pas limitée.
		\subsection{Objectifs}
			\p{Objectif premiers}
				Il faut réaliser une IA mais il faut aussi pouvoir tester cette dernière pour évaluer sa performance. Pour pouvoir tester l'IA il faut que le programme propose de jouer en Humains contre IA mais aussi en IA contre IA. Pour cela nous avons fixé les objectifs suivants:
				\begin{itemize}
					\item Implémenter la logique d'un jeu de Pogo
					\item Réaliser une IA capable de jouer au jeu
					\item Réaliser une interface homme-machine pour permettre à l'utilisateur de visualiser le jeu et d'y jouer
				\end{itemize}
			\p{Objectif supplémentaires}
				Nous avons rajouté quelques objectifs supplémentaires afin d'améliorer le programme, de faciliter les tests de l'IA et d'élargir les possibilités de modification de l'IA.
				\begin{itemize}
					\item Permettre de jouer en Humain contre Humain
					\item Permettre de choisir la profondeur d'exploration de l'arbre de l'IA
					\item Proposer plusieurs fonctions d'évaluation d'une situation de jeu et un comparatif de ces dernières
					%\item \color{green} Ajouter une non-IA (mouvements aléatoires)
				\end{itemize}
		\subsection{Notes}
			\p{Détails}
				Les fonctions importantes et liées à l'IA seront expliquées et leur fonctionnement détaillé (souvent par un algorithme). Pour les autres fonctions se référer à la documentation Doxygen du code.
				\begin{upminfo}
					Doxygen est un générateur de documentation sous licence libre, une documentation web peut être généré, pour cela voir \href{http://www.stack.nl/~dimitri/doxygen/}{le site web de Doxygen}.
				\end{upminfo}
			\p{Diagrammes}
				Les diagrammes utilisés pour expliciter les classes sont des diagrammes de classe UML dont les spécifications peuvent être obtenues sur le site \href{http://www.omg.org/spec/UML/}{Object Management Group}
	\section{Réalisation}
		\subsection{Représentation du jeu}
			\subsubsection{Pions}
				\p{Définition}
					Les pions sont représentés par la classe \jclass{Pawn} définit dans \textbf{Pawn.hpp}
				\p{Observation}
					Les pions ont des caractéristiques intrinsèques identiques pour tous comme leur taille, le nombre de points qu'ils valent\ldots Mais ces derniers n'ont qu'une seule caractéristique intrinsèque variable, le joueur auquel ils appartiennent, cette caractéristique n'a que deux valeurs possibles.
				\p{Choix}
					Les pions sont donc représentés par une valeur booléenne:
					\begin{description}
						\item[0] Pion du joueur 1
						\item[1] Pion du joueur 2
					\end{description}
			\subsubsection{Piles de pions}
				\p{Définition}
					Les piles de pions sont représenté par la classe \jclass{PawnStack} définit dans \textbf{PawnStack.hpp}
					\begin{figure}[H]
						\centering
						\includegraphics[width=0.7\textwidth]{figures/PawnStackDiagram}
						\caption{PawnStack: diagramme de classe}
						\label{fig:PawnStackDiagram}
					\end{figure}
				\p{Choix}
					Partant de l'idée qu'un \textbf{bit} (0 ou 1) suffit pour représenter un pion nous avons choisi d'optimiser la mémoire en représentant une pile de pions par un nombre entier (\textit{container}) ou chaque bit qui le compose représente un pion. Le problème qui se pose alors est lorsqu'il n'y a pas de pions puisque chaque valeur possible représente un joueur.
				\p{Taille de la pile}
					Dans notre représentation une pile a un début fixe, le premier élément, mais une fin variable puisque la taille d'une pile est variable. Comme illustré sur la figure \ref{fig:pile_mémoire}, pour définir la taille d'une pile nous avons décidé qu'en partant de la fin (pile complète, 16 pions) que le premier pion au joueur 2 (bit à 1) marquerait la fin de la pile aux pions précédents.
					\begin{figure}[H]
						\begin{center}
							%\begin{tabular}{|c|c|c|c|c|c|c|c|c|c|c|c|c|c|c|c|c|}
							%	\hline
							%	Numéro du bit & 0 & 1 & 2 & 3 & 4 & 5 & 6 & 7 & 8 & 9 & 10 & 11 & 12 & 13 & 14 & 15\\
							%	\hline
							%	Valeur du bit & 0\cellcolor{green} & 1\cellcolor{green} & 0\cellcolor{green} & 1\cellcolor{green} & 1\cellcolor{green} & 0\cellcolor{green} & 0\cellcolor{green} & 1\cellcolor{green} & 0\cellcolor{green} & 1\cellcolor{orange} & 0\cellcolor{red} & 0\cellcolor{red} & 0\cellcolor{red} & 0\cellcolor{red} & 0\cellcolor{red} & 0\cellcolor{red}\\
							%	\hline
							%\end{tabular}
							\begin{tabular}{|c|c|c|c|c|c|c|c|c|c|c|c|c|c|c|c|c|}
								\hline
								Numéro du bit & 15 & 14 & 13 & 12 & 11 & 10 & 9 & 8 & 7 & 6 & 5 & 4 & 3 & 2 & 1 & 0\\
								\hline
								Valeur du bit & 0\cellcolor{red} & 0\cellcolor{red} & 0\cellcolor{red} & 0\cellcolor{red} & 0\cellcolor{red} & 0\cellcolor{red} & 1\cellcolor{orange} & 0\cellcolor{green} & 1\cellcolor{green} & 0\cellcolor{green} & 0\cellcolor{green} & 1\cellcolor{green} & 1\cellcolor{green} & 0\cellcolor{green} & 1\cellcolor{green} & 0\cellcolor{green}\\
								\hline
							\end{tabular}
						\end{center}
						\caption{Pile en mémoire}
						\label{fig:pile_mémoire}
					\end{figure}
					La figure \ref{fig:pile_représentation} illustre la conversion de la représentation mémorielle vers sa signification puis vers l'affichage pour le joueur. La pile dans cet exemple est donc contrôlée par le joueur 1.
					\begin{figure}[H]
						\centering
						\includegraphics[width=\textwidth]{figures/pile}
						\caption{Représentation d'une pile}
						\label{fig:pile_représentation}
					\end{figure}
				\p{Fonction principales}
					La fonction \jfunc{pick} permet de prendre des pions de la pile et la fonction \jfunc{add} permet d'ajouter des pions à la pile, elles sont utilisées pour appliquer les mouvements de pions.
			\subsubsection{Mouvements de pions}
				\p{Définition}
					Les mouvements de pions sont représentés par la structure \jclass{PawnsMove} définit dans \textbf{PawnsMove.hpp}
					\begin{figure}[H]
						\centering
						\includegraphics[width=0.25\textwidth]{figures/PawnsMoveDiagram}
						\caption{PawnsMove: diagramme de classe}
						\label{fig:PawnsMoveDiagram}
					\end{figure}
				\p{Choix}
					Cette structure représente un mouvement de pions, elle est généré par les joueurs (humains ou IA) lorsqu’ils jouent et est appliqué au plateau.
			\subsubsection{Plateau}
				\p{Définition}
					Le plateau est représenté par la classe \jclass{Board} définit dans \textbf{Board.hpp}
					\begin{figure}[H]
						\centering
						\includegraphics[width=0.8\textwidth]{figures/BoardDiagram}
						\caption{Board: diagramme de classe}
						\label{fig:BoardDiagram}
					\end{figure}
				\p{Choix}
					Le plateau est l'équivalent d'un tableau 2 dimensions $width \times height$, dans notre cas $width$ et $height$ auront pour valeur 3.
				\p{Fonction principales}
					La fonction \jfunc{apply} permet d'appliquer un mouvement de pions et ainsi de jouer. La fonction \jfunc{controlledStacks} permet de connaitre le nombre de piles contrôlées par un joueur et donc de savoir si un joueur a gagné, si la fonction retourne 0 pour un joueur, alors ce dernier ne contrôle plus de pile et a donc perdu.
		\subsection{Structure des joueurs}
			\subsubsection{Cas général}
				\p{Définition}
					Un joueur est représenté par la classe \jclass{Player} définit dans \textbf{Player.hpp}
					\begin{figure}[H]
						\centering
						\includegraphics[width=0.8\textwidth]{figures/PlayerDiagram}
						\caption{Player: diagramme de classe}
						\label{fig:PlayerDiagram}
					\end{figure}
				\p{Utilisation}
					Pour pouvoir être utilisé avec le \jclass{GameManager} et représenter un joueur, il suffit d'hériter de la classe \jclass{Player} et d'implémenter la méthode \jfunc{chooseMove} qui en fonction du plateau qui lui est passé en paramètre choisit le coup a jouer. De l'implémentation de cette méthode dépendra le type de joueur, ont été implémentées: Humain, IA avec MinMax et IA avec AlphaBeta.
					\begin{figure}[H]
						\centering
						\includegraphics[width=0.8\textwidth]{figures/PlayersDiagram}
						\caption{Players: diagramme de classe}
						\label{fig:PlayersDiagram}
					\end{figure}
			\subsubsection{Joueur humain}
				\p{Définition}
					Un joueur humain est représenté par la classe \jclass{HumanPlayer} définie dans \textbf{HumanPlayer.hpp}, son constructeur prend en paramètre le pion (\jclass{Pawn}) utilisé par le joueur ainsi que l'interface utilisateur (\jclass{UI}) du jeu.
				\p{Fonctionnement}
					Lors de l'appel de la fonction \jfunc{chooseMove} de la classe \jclass{HumanPlayer}, la fonction \jfunc{chooseMove} de la classe \jclass{UI} est appelée pour permettre à l'utilisateur de choisir le mouvement.
			\subsubsection{IA avec MinMax}
				\p{Définition}
					Un joueur IA utilisant l'algorithme MinMax est représenté par la classe \jclass{MinMaxAIPlayer} définie dans \textbf{MinMaxAIPlayer.hpp}, son constructeur prend en paramètre le pion (\jclass{Pawn}) utilisé par le joueur, la profondeur de l'arbre à parcourir et la fonction d’évaluation du plateau à utiliser.
				\p{Fonctionnement}
					Lors de l'appel de la fonction \jfunc{chooseMove}, l'algorithme du MinMax suivant est utilisé:
					\lstinputalgo[caption=MinMax]{algorithms/MinMax.algo}
					Comme l'algorithme n'a pas de ``mémoire'' du mouvement et retourne simplement l'évaluation minimale/maximale, le premier appel diffère. Il est réalisé par la fonction \jfunc{chooseMove} qui réalise le même algorithme mais en sauvegardant le meilleur mouvement rencontré pour le retourner et donc le jouer.\\
					La génération des coups utilisée lors du \texttt{<for each possible move do>} est détaillée dans la section \ref{moveGeneration}.
			\subsubsection{IA avec AlphaBeta}
				\p{Définition}
					Un joueur IA utilisant l'algorithme AlphaBeta est représenté par la classe \jclass{AlphaBetaAIPlayer} définit dans \textbf{AlphaBetaAIPlayer.hpp}, son constructeur prend en paramètre le pion (\jclass{Pawn}) utilisé par le joueur, la profondeur de l'arbre à parcourir et la fonction d’évaluation du plateau à utiliser.
				\p{Fonctionnement}
					Lors de l'appel de la fonction \jfunc{chooseMove}, l'algorithme du AlphaBeta suivant est utilisé:
					\lstinputalgo[caption=AlphaBeta]{algorithms/AlphaBeta.algo}
					Pour la même raison que avec le MinMax, le premier appel est réalisé par la fonction \jfunc{chooseMove} qui réalise le même algorithme mais en sauvegardant le meilleur mouvement rencontré pour le retourner et donc le jouer.\\
					La génération des coups utilisée lors du \texttt{<for each possible move do>} est détaillée dans la section suivante.
			\subsubsection{Génération des coups}\label{moveGeneration}
				\paragraph{}
					Pour générer les coups, nous devions modéliser les règles de déplacement. Nous avons d'abord envisagé, pour déterminer tous les déplacements possibles à partir d'une pile, d'utiliser un algorithme récursif, qui déterminerait les piles d'arrivés par diffusion à partir de la pile de départ.\\
					Nous avons finalement opté pour un calcul de distance de Manhattan défini comme:
					\[d((x1,y1),(x2,y2)) = |x2 - x1| + |y2 -y1|\]
				\paragraph{}
					En effet, on observe que l'on peut bouger $p$ pions d'une pile $i$ vers une pile $j$ si l'égalité $d(pile_{i}, pile_{j}) = p$ est vérifiée. Il faut cependant rajouter la possibilité du déplacement de 3 pions d'une distance de 1 case, due aux déplacements en coudes.
				\paragraph{}
					Voici un exemple illustrant tous les déplacements possibles à partir d'une pile, en utilisant les distances de Manhattan (La pile de départ est encadrée en vert).
					\begin{figure}[H]
						\centering
						\includegraphics[width=0.6\textwidth]{figures/GenerationCoup}
						\caption{Génération des coups}
					\end{figure}
			\subsubsection{Fonction d'évaluation}
				\p{Choix}
					Afin d'évaluer la partie nous comptons pour un joueur  le nombre de piles qu’il contrôle, que nous divisons par le nombre total de piles présentes sur le jeu. Notre évaluation représente donc le pourcentage de piles que nous contrôlons. Eval = 0 représente donc une défaite pour le joueur qui évalue car il ne possède aucune pile et Eval = 1 une victoire car il les contrôle toutes. On a donc $Eval \in [0,1]$.\\
					De façon formelle :
					\[
					x_{i} = \left\{
					\begin{array}{ll}
						1 \mbox{ si  on contrôle la pile de la case i} \\
						0  \mbox{ sinon}
					\end{array}
					\right.
					\]
					et $n$ le nombre de piles présentes sur le plateau
					\[Eval = \frac{1}{n}  \sum_{i = 1}^{9} x_{i}\]
				\p{Exemple}
					Dans la figure \ref{fig:EvalExample}, le joueur violet contrôle 4 piles, et le nombre de piles sur le plateau est de 7, ce qui nous donne $Eval = \frac{4}{7} \simeq 0,571$.
					\begin{figure}[H]
						\centering
						\includegraphics[width=0.6\textwidth]{figures/Eval.png}
						\caption{Exemple d'évaluation}
						\label{fig:EvalExample}
					\end{figure}
				\p{Autres possibilités}
					D'autres fonctions d'évaluation sont possibles, nous avons implémenté deux d'entre elles:
					\begin{itemize}
						\item La fraction des pions contrôlés, semblable à celle que nous utilisons mais en se basant sur les pions contrôlés au lieu des piles controlées.
						\item La différence entre le nombre de piles du joueur A et du joueur B, l’évaluation retourne 0 lorsqu'elle évalue le plateau pour un joueur qui a perdu
					\end{itemize}
					Pour les utiliser il suffit de changer la fonction d'évaluation passée au constructeur des IAs.
		\subsection{Interface utilisateur}
			\subsubsection{Fonctionnalités}
				L'interface graphique est entièrement réalisée en console avec un style ``graphique''. Il y a deux interfaces principales, les menus et messages (d'information, d'erreur\ldots) et le plateau du jeu de Pogo.
				\p{Menus}
					Avant de commencer une partie plusieurs menus se suivent:
					\begin{itemize}
						\item Menus principal: choix du mode de jeu
							\begin{itemize}
								\item Humain contre Humain
								\item Humain contre IA
								\item IA contre IA
							\end{itemize}
						\item Si jeu avec IA (une ou deux): Menu de choix de la profondeur d'exploration de l'arbre de/des IA
						\item Menu de choix de l'ordre de jeu
					\end{itemize}
				\p{Plateau de jeu}
					Le plateau de jeu en 3 $\times$ 3 fait une taille de 98 caractères de large et 53 caractères de haut, si la console n'est pas suffisamment grande pour afficher le plateau un message demandant d'agrandir la console apparaitra.
					\begin{itemize}
						\item Dans le cas d'une utilisation sous Windows si l'écran ne permet pas d'étendre suffisamment la console on peut faire:\\
							\textit{clique droit sur la barre supérieure > Propriétés > Police > Taille}\\
							et réduire la taille de la police pour augmenter la largeur et hauteur en caractères de la console pour une même taille et permettre l'affichage du plateau.
						\item Dans le cas d'une version de Windows < 8, la largeur de la console est limitée a une valeur par défaut il faut donc changer la taille du buffer pour un minimum de 98 caractères de large (et 53 caractères de haut) toujours avec:\\
							\textit{clique droit sur la barre supérieure > Propriétés}
					\end{itemize}
					Lors du tour d'un joueur humain, le plateau permet de choisir un mouvement avec la séquence de choix suivante:
					\begin{itemize}
						\item Choix de la pile de départ
						\item Choix du nombre de pions à prendre
						\item Choix de la pile d'arrivé
					\end{itemize}
					Lors du choix de la pile de départ et d'arrivé, les choix valides seront affichés en vert.
			\subsubsection{Utilisation}
				\p{Menus}
					Les menus simples s'utilisent avec les contrôles suivants:
					\begin{description}
						\item[Page précédente] Premier choix
						\item[Page suivante] Dernier choix
						\item[Flèche du haut] Choix précédent
						\item[Flèche du bas] Choix suivant
						\item[Entré] Valider le choix
						\item[Échap] Quitter (pas toujours possible)
					\end{description}
					Les menus à options s'utilisent avec les contrôles des menus simples avec en plus que les contrôles suivants:
					\begin{description}
						\item[Début] Première valeur de l'option
						\item[Fin] Dernière valeur de l'option
						\item[Flèche de gauche] Valeur précédente de l'option
						\item[Flèche de droite] Valeur suivante de l'option
					\end{description}
				\p{Plateau de jeu}
					Lors de la sélection d'une pile, les contrôles suivants sont utilisables:
					\begin{description}
						\item[Flèche du haut] Bouger la sélection en haut
						\item[Flèche du bas] Bouger la sélection en bas
						\item[Flèche de gauche] Bouger la sélection à gauche
						\item[Flèche de droite] Bouger la sélection à droite
						\item[Entré] Valider le choix
						\item[Échap] Quitter la partie
					\end{description}
					Lors de la sélection du nombre de pions à prendre dans une pile:
					\begin{description}
						\item[Flèche du haut] Diminuer le nombre de pions pris
						\item[Flèche du bas] Augmenter le nombre de pions pris
						\item[Entré] Valider le choix
						\item[Échap] Quitter la partie
					\end{description}
			\subsubsection{Réalisation}
				\begin{figure}[H]
					\centering
					\includegraphics[width=0.7\textwidth]{figures/UIDiagram}
					\caption{UI et ConsoleUI: diagramme de classe}
					\label{fig:UIDiagram}
				\end{figure}
				L'interface utilisateur du projet est constituée de deux classes:
				\p{UI}
					\texttt{UI} est l'interface (classe purement abstraite) à implémenter pour réaliser une interface homme machine utilisable dans le projet. Elle a 3 paramètres de template: le type de pile utilisé, la largeur du plateau et sa hauteur. Elle possède des fonctions pour afficher le plateau et la victoire mais aussi pour permettre au joueur humain de choisir un mouvement.
				\p{ConsoleUI}
					\texttt{ConsoleUI} est l'implémentation qu'utilise le projet, elle est entièrement en console mais emprunte beaucoup aux interfaces graphiques, elle a été réalisée avec la librairie C, compatible C++ \textbf{ConsoleControl} développée par Maxime Pinard, membre du groupe du projet. Le projet utilise la version 0.2 de la librairie.
					\begin{figure}[H]
						\centering
						\includegraphics[width=0.7\textwidth]{figures/ConsoleUI}
						\caption{ConsoleUI: affichage en console du plateau de jeu}
						\label{fig:ConsoleUI}
					\end{figure}
				\p{ConsoleControl}
					Fonctionnalités:
					\begin{itemize}
						\item Obtention d'informations sur la console (largeur, hauteur\ldots)
						\item Positionnement du curseur
						\item Changement des couleurs d'arrière-plan et de premier plan
						\item Gestion des inputs
						\item Dessin géométrique (lignes, rectangle)
						\item Interface utilisateur:
							\begin{itemize}
								\item Menu
								\item Menu d'options
								\item Messages
							\end{itemize}
						\item \ldots
					\end{itemize}
					Avantages:
					\begin{itemize}
						\item Multiplateforme
						\item Pas de dépendances
						\item Utilisé en sous module compilé en même temps que le projet
					\end{itemize}
					Pour plus d'informations, voir \href{https://github.com/pinam45/ConsoleControl}{la page Github de la librairie}.
				\p{Résultat}
					L'interface du jeu est multiplateforme, dessinée en console, principalement en couleur mais aussi avec des caractères. Nous avons choisi des couleurs différentes du jeu original de Pogo pour plus de lisibilité dans la console mais les couleurs et les caractères utilisés sont tous des champs privés de \textit{ConsoleUI} et sont donc configurables dans le fichier \textbf{ConsoleUI.hpp}.
	\section{Compilation}
		\subsection{Informations}
			\p{Options}
				La compilation se fait avec avec le standard \textbf{C++14}, un compilateur suffisamment à jour (gcc à partir de 4.9) est donc nécessaire.\\
				Deux modes de compilation sont possibles:
				\begin{description}
					\item[Debug] ajoute les informations de debug et active le logger
					\item[Release] active les optimisations de performance du compilateur
				\end{description}
			\p{Tests}
				La compilation a été testée dans les environnements suivants:
				\begin{center}
					\begin{tabular}{|c|c|c|c|}
						\hline
						OS & système de compilation & compilateur & résultat\\
						\hline
						Debian & CMake / Make & gcc 5.3 & OK\cellcolor{green}\\
						\hline
						Debian & CMake / Make & gcc 6.2 & OK\cellcolor{green}\\
						\hline
						Debian & CMake / Make & gcc 6.3 & OK\cellcolor{green}\\
						\hline
						Windows 10 & CMake / Make (Mingw-w64 5.0.2) & Mingw-gcc 5.3.0 & OK\cellcolor{green}\\
						\hline
						Windows 10 & CodeBlocks 13.12 (Mingw embarqué) & Mingw-gcc 4.7.1 & erreur\cellcolor{red}\\
						\hline
						Windows 10 & CodeBlocks 13.12 (Mingw-w64 5.0.2) & Mingw-gcc 5.3.0 & OK\cellcolor{green}\\
						\hline
						Windows 10 & CodeBlocks 16.01 (Mingw embarqué) & Mingw-gcc 5.3.0 & erreur\cellcolor{red}\\
						\hline
						Windows 10 & CodeBlocks 16.01 (Mingw-w64 5.0.2) & Mingw-gcc 5.3.0 & OK\cellcolor{green}\\
						\hline
					\end{tabular}
				\end{center}
			\p{Flags utilisé (gcc)}
				$\bullet$ Flags de warning:
				\begin{lstlisting}[breaklines=true,breakatwhitespace=true,breakindent=0pt,columns=fixed,keepspaces=true,frame=single,basicstyle=\footnotesize\sffamily]
-pedantic -pedantic-errors -Wall -Wcast-align -Wcast-qual -Wconversion -Wdisabled-optimization -Wdouble-promotion -Wextra -Wformat -Winit-self -Winvalid-pch -Wlogical-op -Wmain -Wmissing-declarations -Wmissing-include-dirs -Wold-style-cast -Woverloaded-virtual -Wpointer-arith -Wredundant-decls -Wshadow -Wswitch-default -Wswitch-enum -Wundef -Wuninitialized -Wunreachable-code -Wwrite-strings -Weffc++\end{lstlisting}
				Pour plus d'information voir les \href{https://gcc.gnu.org/onlinedocs/gcc/Warning-Options.html}{options de warning gcc}.\\
				$\bullet$ Flag en debug:
				\begin{lstlisting}[breaklines=true,breakatwhitespace=true,breakindent=0pt,columns=fixed,keepspaces=true,frame=single,basicstyle=\footnotesize\sffamily]
-g -DDEBUG -DLOGGER_ENABLED\end{lstlisting}
				Le fait de définir \textit{LOGGER\_ENABLED} active le logger.\\
				$\bullet$ Flag en release:
				\begin{lstlisting}[breaklines=true,breakatwhitespace=true,breakindent=0pt,columns=fixed,keepspaces=true,frame=single,basicstyle=\footnotesize\sffamily]
-s -Os\end{lstlisting}
			\p{Compilation}
				Pour compiler le programme il est nécessaire de compiler d'abord la librairie \textit{ConsoleControl} ou bien de télécharger la dernière release puis de la linker à la compilation du projet. La librairie étant fournie avec le projet plusieurs systèmes de compilation qui automatise cette tache ainsi que la compilation du projet sont disponibles, il est recommandé de les utiliser.
			\p{Systèmes de compilation}
				Deux systèmes de compilations sont mis à disposition:
				\begin{description}
					\item[Make] Fichier \textit{Makefile}
					\item[CMake] Fichier \textit{CMakeLists.txt}
				\end{description}
		\subsection{Make}
			Make s'utilise avec un \textit{Makefile}, pour plus d'informations sur Make voir \href{https://www.gnu.org/software/make/}{le site de gnu}\\
			Le \textit{Makefile} se configure grace aux variables définies dans les deux premières sections, la configuration de base fournie est celle pour un OS Linux standard utilisant les commandes shell et gcc.
			\p{Compilation}
				Plusieurs cibles sont définies:
				\begin{description}
					\item[help] Affiches les cibles définit et leur effet
					\item[silent] Cible par défaut, équivalent a \texttt{make --silent all}
					\item[all] Compile le projet
					\item[debug] Compile le projet en debug
					\item[clean] Supprime tous les fichiers et dossiers générés par le \textit{Makefile}
				\end{description}
				Pour appeler une cible:
				\begin{lstlisting}[breaklines=true,breakatwhitespace=true,breakindent=0pt,columns=fixed,keepspaces=true,frame=single,basicstyle=\footnotesize\sffamily]
$ make <cible>\end{lstlisting}
				Pour simplement compiler le projet:
				\begin{lstlisting}[breaklines=true,breakatwhitespace=true,breakindent=0pt,columns=fixed,keepspaces=true,frame=single,basicstyle=\footnotesize\sffamily]
$ make\end{lstlisting}
		\subsection{CMake}
			CMake est un moteur de production multi plate-forme, il est capable de générer des \textit{Makefile} mais aussi des projet Visual Studio et Code::Blocks, pour plus d'information sur CMake voir \href{https://cmake.org/}{le site de CMake}.\\
			\p{Compilation}
				Il existe de très nombreuses façons d'utiliser CMake, la plus courante est de se placer dans le dossier du projet puis:
				\begin{lstlisting}[breaklines=true,breakatwhitespace=true,breakindent=0pt,columns=fixed,keepspaces=true,frame=single,basicstyle=\footnotesize\sffamily]
$ mkdir build
$ cd build
$ cmake ..
$ make \end{lstlisting}
	\section{Résultats}
		\subsection{Performances AlphaBeta vs MinMax}
			\paragraph{}
				Nous avons d'abord implémenté une IA reposant sur le MinMax. Nous avons testé jusqu'à quelle profondeur le temps de réponse de l'IA restait raisonnable pour des parties rapides, soit une réponse en quelques secondes (temps < 10 secondes).\\
				La profondeur maximale répondant à ce critère de temps de réponse avec le MinMax est de 5. En élaguant avec l'AlphaBeta, nous sommes parvenus à obtenir une profondeur maximale de 7. Nous pouvons expliquer cette amélioration par la symétrie des coups possibles, et du petit nombre de valeurs différentes que peut prendre notre évaluation.
		\subsection{Comportements de l'IA}
			\paragraph{}
				L'IA dès le 1er tour à profondeur 1 et 2 va contrôler directement une pile adverse ou contrôler une case vide. Ce résultat est assez évident, compte tenu de notre évaluation qui augmente lorsque l'on contrôle une nouvelle pile.\\
				A profondeur plus élevée, L'IA joue plus prudemment, elle a tendance en début de partie à empiler des pions sur ces propres piles. En regardant plus loin dans la partie, L'IA se rend compte que capturer directement les piles adverses est désavantageux, car elles pourront rapidement être reprises par l'adversaire, ce qui laisserait dominer son adversaire.
			\paragraph{}
				L'IA a tendance à ne pas gagner directement dès qu'il le peut, car il peut voir sur plusieurs coups à l'avance qu'il peut gagner autrement qu'à ce tour ci. On aurait pu facilement forcer l'IA à gagner dès qu'il le peut en prenant en compte la profondeur des coups dans l'évaluation, mais nous avons décidé de ne pas le faire, car dans tous les cas L'IA gagnerait et cela entrainerait des coûts de calculs supplémentaires.
			\paragraph{}
				L'IA répond toujours aux coups de la même façon, bien qu'il y ait des coups avec évaluation identiques, nous n'avons pas choisi de sélectionner au hasard un coup parmi les meilleurs coups, car on optimise en élaguant l'arbre avec l'AlphaBeta. On aurait pu le faire avec le MinMax, mais cela ne changerait pas l'efficacité de L'IA, ça aurait juste permis aux joueurs humains de jouer des parties plus diversifiées.
		\subsection{IA contre IA}
			\paragraph{}
				En lançant une IA avec l'évaluation prenant en compte le pourcentage de piles que nous contrôlons contre l'évaluation se basant sur la fraction des pions contrôlés, on observe plus de victoire du côté de l'IA avec l'évaluation du pourcentage de piles.
			\paragraph{}
				On a aussi remarqué un nouveau comportement, les IA répètent les mêmes coups, et la partie boucle car les IAs ne "veulent" pas prendre de risque, car elles savent que faire autre d'autres coups pourraient les faire perdre.
		\subsection{IA contre Humain}
			\p{Kadir:}
				En choisissant des profondeurs peu élevées (jusqu'à 3), il est assez simple de battre L'IA en jouant prudemment, et en jouant de sorte à créer le plus de pile possible en notre faveur. Ensuite il devient plus compliqué de les battre, à profondeur 4/5, en imitant le coup de départ de l'IA, j'ai réussi à gagner, il suffit d'essayer plusieurs fois et retenir les bon/mauvais coups. Je n'ai pas réussi à battre l'IA au-delà de la profondeur 5.
\end{document}
