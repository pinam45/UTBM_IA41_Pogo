\documentclass[report, backcover, french, nodocumentinfo]{upmethodology-document}
\include{packages}
% For more information about UPmethodology: https://www.ctan.org/pkg/upmethodology

%=======================================================================================================
%=============================================== Informations ==========================================
%=======================================================================================================

%%% Document Information and Declaration
\declaredocument{Jeu tactique: Pogo}{Projet d'IA41}{UTBM\_IA41\_Pogo\_P17}

%%% Abstract and Key-words
\setdocabstract[french]{Projet UTBM de l'UV IA41 du semestre de printemps 2017}
\setdockeywords[french]{UTBM, IA41, Pogo}

%%% Document Authors and Validators
\addauthorvalidator*[julien.barbier@utbm.fr]{Julien}{Barbier}{Étudiant en INFO02}
\addauthorvalidator*[kadir.ercin@utbm.fr]{Kadir}{Ercin}{Étudiant en INFO02}
\addauthorvalidator*[maxime.pinard@utbm.fr]{Maxime}{Pinard}{Étudiant en INFO02}

%%% Informed People
\addinformed*[fabrice.lauri@utbm.fr]{Fabrice}{Lauri}{Professeur de l'UV IA41}

%%% Copyright and Publication Information
\setcopyrighter{Julien Barbier, Kadir Ercin et Maxime Pinard}
\setpublisher{Julien Barbier, Kadir Ercin et Maxime Pinard}
\setprintingaddress{France}

%%% Version
\incversion{\makedate{\the\day}{\the\month}{\the\year}}{Initial version.}{\upmpublic}

%=======================================================================================================
%================================================== Configs ============================================
%=======================================================================================================

% Change Front Page Layout
%\setfrontcover{modern} % modern or classic

% Change Illustration Picture
%\setfrontillustration[1.3]{figures/figure}

% Source code formatting
\upmcodelang{cpp} % uml, java or cpp

% Prevent page breaks in paragraphs
\predisplaypenalty=1000
\postdisplaypenalty=1000
\clubpenalty=1000

% Minimal space required in the bottom margin not to move the title on the next page
%\renewcommand{\bottomtitlespace}{.1\textheight}

% Links config, especialy for the table of contents
\hypersetup{
    colorlinks=true,
    linkcolor=black,
    urlcolor=blue,
    linktoc=all
}

% French language config
%\frenchbsetup{StandardLayout=true,ReduceListSpacing=false,CompactItemize=false}

%=======================================================================================================
%================================================= Functions ===========================================
%=======================================================================================================

%Paragraph with line break
\newcommand{\p}[1]{\paragraph{#1\\}}

% Function to print a warning sign
\newcommand{\dangersign}[1][2.5ex]
	{\renewcommand{\stacktype}{L}
		{\scaleto{\stackon[1pt]{\color{red}$\triangle$}{\fontsize{4pt}{4pt}\selectfont !}}{#1}}}

% Definition of some dt/dx/dy shortcuts for integrals
\newcommand{\dt}
{\;\mathrm{d}\,t}

\newcommand{\dx}
{\;\mathrm{d}\,x}

\newcommand{\dy}
{\;\mathrm{d}\,y}

% Definition of \Witem for 'itemize' environment with a warning sign
\newcommand{\Witem}
{\item[\dangersign{}]}

% Definition of a Max function shortcut
\newcommand{\Max}[2][ ]
{\underset{#1}{\text{Max}}\,#2}

% Definition of a Min function shortcut
\newcommand{\Min}[2][ ]
{\underset{#1}{\text{Min}}\,#2}


\begin{document}
	\upmdocumentsummary{}
	\upmdocumentauthors{}
	%\upmdocumentvalidators{}
	\upmdocumentinformedpeople{}
	\upmpublicationpage{}
	\tableofcontents{}
	%\listoffigures{}
	\newpage{}

	\chapter{Sujet, règles du jeu}
		\section{Présentation}
		\section{Règles}
		\section{Objectif}
	\chapter{Réalisation}
		\section{Représentation du jeu}
			\subsection{pions}
			\subsection{Piles de pions}
			\subsection{Plateau}
		\section{Structure de l'IA}
			\subsection{MinMax}
			\subsection{AlphaBeta}
		\section{Interface utilisateur}
			\subsection{Réalisation}
			\subsection{Utilisation}
	\chapter{Résultats}
		\section{IA contre IA}
		\section{IA contre Humain}
%\begin{upmcaution}
%	This is an example of a caution message. This text must be rendered with enough height (usually 2 lines of text) to avoid intersection between %the caution icon and the box frame.
%\end{upmcaution}
%\begin{upminfo}
%	This is an example of an information message. This text must be rendered with enough height (usually 2 lines of text) to avoid intersection %between the caution icon and the box frame.
%\end{upminfo}
%\begin{upmquestion}
%	This is an example of a question message. This text must be rendered with enough height (usually 2 lines of text) to avoid intersection between %the caution icon and the box frame.
%\end{upmquestion}
%\begin{figure}[h!]
%	\centering
%	\includegraphics[width=0.5\textwidth]{figures/sample} % or <figures/sample.jpg>
%	\caption{sample figure}
%	\label{fig:sampleFigure}
%\end{figure}
\end{document}
